\documentclass[twocolumn,11pt]{IEEEtran}
\usepackage[verbose,expansion=alltext,stretch=50]{microtype}

\title{Optimizing Scheduling Policies For Heterogeneous Distributed Systems}

\author{
   \small{Team 10}
   
   \begin{tabular}{c| c| c| c}
       Mahmoud Adas & Evram Youssef & Mohamed Shawky & Remonda Talaat\\
       \texttt{\small{SEC 2, B.N 21}} & \texttt{\small{SEC 1, B.N 9}} & \texttt{\small{SEC 2, B.N 16}} & \texttt{\small{SEC 1, B.N 20}}
   \end{tabular}%
   
   \texttt{\small{\{mido3ds, evramyousef, mohamedshawky911, remondatalaat21\}@gmail.com}}
}%

\markboth{Cairo Uni. CMP Dep. | OS Research | Phase 2 | Team 10 | \today}{Shell}

\begin{document}
    \maketitle

    \begin{abstract}
        This is phase 2 of the research project.
        This paper describes the problem and explains why it is important.
        It also cites 3 papers with their critiques.
    \end{abstract}

    \section{Introduction}
    \IEEEPARstart{a}{} \emph{Heterogeneous System} is a system with multiple processors with some of them based on different instruction-set architecture (ISA) or micro-architecture. Usually, each processor excels at some aspect and performs poorly at another, and by combining them, we provide the best of each world. 
    Heterogeneous systems are more complex than homogeneous ones, because processors have different architectures. That means they need more complex scheduling policies to utilize the different properties efficiently.
    
    \section{The Problem}
    Our goal is to explore the possibility of providing more optimal scheduler policy for heterogeneous systems through 
    heuristic parameter optimization methods like different machine learning and/or genetic algorithms.
    We plan to build a simulation for some popular heterogeneous system, and integrate our optimization technique to 
    find better policy for different scenarios.
    
    \section{Why is it important?}
    Because heterogeneous systems are relatively new concept, they lack good support from OS developers, 
    and thus their schedulers policies still need more optimizations. 
    Recent findings show that they, the heterogeneous systems, can achieve lower energy delay product over homogeneous systems as much as 21\%-23\%. We assume that with more efficient scheduling they can achieve lower power consumption, and then they can become more economical.
    
    \section{References}
    \subsection*{[1] Ashish Venkat and Dean M Tullsen. Harnessing isa diversity: Design  of  a  heterogeneous-isa  chip  multiprocessor. In 2014 ACM/IEEE 41st International Symposium on Computer Architecture(ISCA), pages 121–132. IEEE, 2014.}
    \begin{itemize}
        \item Why this paper: To understand heterogeneous systems structures.
        \item Paper research problem: Which one is more power effective, heterogeneous or homogeneous systems?
        \item Paper goal: prove heterogeneous systems are more power efficient
        \item Conclusion: heterogeneous systems improves energy efficiency over the most efficient single-ISA design by 23\%.
    \end{itemize} 
    
     \subsection*{[2] Padole, Dr. Mamta \& Shah, Ankit. (2018). Comparative Study of Scheduling Algorithms in Heterogeneous Distributed Computing Systems.}
    \begin{itemize}
        \item Why this paper: To understand the concepts and concerns of scheduling in heterogeneous systems, as well as the recent algorithms.
        \item Paper research problem: The recent scheduling algorithms for heterogeneous systems and their specifications.
        \item Paper goal: The paper offers a hierarchical view of the static and dynamic scheduling algorithms found by then and gives an brief overview for each one. 
        \item Conclusion: Most of the current scheduling algorithms can be further improved in various aspects, moreover static algorithms can be re-implemented to be dynamic.
    \end{itemize} 
    
    \subsection*{[3] Arabnejad, Hamid \& Barbosa, Jorge. (2014). List Scheduling Algorithm for Heterogeneous Systems by an Optimistic Cost Table. IEEE Transactions on Parallel and Distributed Systems.}
    \begin{itemize}
        \item Why this paper: To get a sense of predictive scheduling in heterogeneous systems, as we will be using predictive models in our future work.
        \item Paper research problem: Task scheduling in heterogeneous systems based on Optimistic Cost Table (OCT).
        \item Paper goal: Achieve the same time complexity as the state-of-art methods, but with higher overall scheduling performance.
        \item Conclusion: Predictive-Earliest-Finish-Time (PEFT) is outperforms Heterogeneous-Earliest-Finish-Time (HEFT) with the same time complexity.
    \end{itemize} 
\end{document}


