\IEEEPARstart{T}{he} continuous evolution of computing power and growth of widely-distributed applications and \emph{Internet of Things} (IoT) have driven our needs for better resource allocation methods. 
The performance of any system critically depends on the algorithms used to schedule tasks on its resources. These systems contain limited resources, that should be allocated properly, in order to perform the required tasks with maximum efficiency avoiding starvation and resource exhaustion. 
The general objective of any scheduling technique is to get the best possible performance with a reasonable resource utilization. \\ 

Recently, the heterogeneity of the computational systems has increased tremendously, due to the nature of the applications and the great advances in IoT and distributed systems. A \emph{heterogeneous system} can be defined as a range of resources, different in underlying architecture, targeting different types of computational tasks. The usage of a heterogeneous system has proven to be very efficient in increasing the system performance and reducing the overall power consumption. Recent studies \cite{venkat2014harnessing} show that \emph{heterogeneous system} can outperform the best homogeneous systems by as much as 21\%, with 23\% energy savings and
a reduction of 32\% in Energy Delay Product. These numbers can be improved even more with better task scheduling and resource allocation methods. \\

However, this comes with the problem of complex task scheduling. The task scheduling problem for a heterogeneous computing system is more complex than that for a homogeneous system, because of the different execution rates of processors and different communication rates among different processors, which add extra layers of complexity to the problem. \\

The parameters of such problem make it very suitable for various optimization techniques. Recursive optimization techniques are very efficient in solving many optimization problem. Previous work has investigated the usage of techniques such as \emph{genetic algorithms} \cite{article2} to solve the task scheduling problem. Also, the usage of a progressive decision making techniques such as \emph{Reinforcement Learning} \cite{ORHEAN2018292} for task scheduling in heterogeneous systems has been investigated. \\

In this paper, we investigate the ability of advanced predictive models to learn a representation of static local task scheduling problem. This learned representation will be used to approximate the scheduling performance of other scheduling techniques. \emph{Neural Networks} are our best candidate for this task. Neural networks are being used in many fields replacing traditional methods, such as image classification \cite{10.1145/3065386}, object detection \cite{ren2015faster}, neural language modelling \cite{ren2015faster}, image generation \cite{karras2019analyzing} and others. They have proven to be more efficient than other traditional methods. The complexity of the neural networks comes in the development phases, where the network is designed and trained. However, in inference phase, neural networks can achieve very accurate results at high speed. Recent work has been done on optimizing neural network inference through various techniques, such as quantization \cite{choukroun2019lowbit} and pruning \cite{yeom2019pruning}. \\

The recent advances in \emph{Deep Learning} research is considered in designing and training our scheduling network. \emph{Heterogeneous Earliest Time First} (HEFT) \cite{993206} is used as a target, where the network learns to approximate its performance. Also, we include a genetic algorithm experiment for the same problem for comparison. We will discuss our approach to the problem and its limitations.