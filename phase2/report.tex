\documentclass[twocolumn,11pt]{IEEEtran}
\usepackage[verbose,expansion=alltext,stretch=50]{microtype}

\title{Optimizing Scheduling Policies For Heterogeneous Distributed Systems}

\author{
   \begin{tabular}{c c c c}
       Mahmoud Adas & Evram Yousef & Mohamed Shawky & Remonda Talaat\\
       \texttt{\small{SEC 2, B.N 21}} & \texttt{\small{SEC 1, B.N 9}} & \texttt{\small{SEC 2, B.N 16}} & \texttt{\small{SEC 1, B.N 20}}
   \end{tabular}%
   \\
   \texttt{\small{\{mido3ds, evramyousef, mohamedshawky911, remondatalaat21\}@gmail.com}}
}%

\markboth{Cairo Uni. CMP Dep. OS, Research Phase 2, \today}{Shell}

\begin{document}
    \maketitle

    \begin{abstract}
        This is phase 2 of the research project.
        This paper describes the problem and explains why it is important.
        It also cites the most 3 papers with their critiques.
    \end{abstract}

    \section{Introduction}
    \IEEEPARstart{a}{} \emph{Heterogeneous System} is a system with multiple processors with some of them based on different instruction-set architecture (ISA) or micro-architecture. Usually, each processor excels at some aspect and performs poorly at another, and by combining them, we provide the best of each world. 
    Heterogeneous systems are more complex than homogeneous ones, because processors have different architectures. That means they need more complex scheduling policies to utilize the different properties efficiently.
    
    \section{The Problem}
    Our goal is to explore the possibility of providing more optimal scheduler policy for heterogeneous systems through 
    heuristic parameter optimization methods like different machine learning and/or genetic algorithms.
    We plan to build a simulation for some popular heterogeneous system, and integrate our optimization technique to 
    find better policy for different scenarios.
    
    \section{Why is it important?}
    Because heterogeneous systems are relatively new concept, they lack good support from OS developers, 
    and thus their schedulers policies still need more optimizations. 
    Recent findings show that they, the heterogeneous systems, can achieve lower energy delay product over homogeneous systems as much as 21\%-23\%. We assume that with more efficient scheduling they can achieve lower power consumption, and then they can become more economical.
    
    \section{References}
    \subsection*{[1] Alexandru Iulian Orhean,  Florin  Pop, and Ioan Raicu. New scheduling approach using reinforcement learning for heterogeneous distributed systems. Journal of Parallel and Distributed Computing, 117:292–302, 2018.}
    \begin{itemize}
        \item Why this paper: So we get insights into how to use machine learning in achieving better scheduling policies.
        \item Paper research problem: Heterogeneous systems task scheduling problem.
        \item Paper goal: Determining a more efficient scheduling policy for heterogeneous systems.
        \item Tools: BURLAP library, Java RMI API, remote allocated schedulers and WorkflowSim.
        \item Conclusion: The paper proposed a platform of scheduling solutions as a service based on machine learning agents. 
        And found out that reinforcement learning has the limitation of with more nodes the system was incapable of learning optimal policy.
    \end{itemize}

    \subsection*{[2] Rakesh Kumar, Er Rajiv Kumar, Er Sanjeev Gill, and Er Ashwani Kaushik. Genetic algorithm approach to operating system process scheduling problem. International journal of Engineering science and Technology, 2(9):4247–4252, 2010.}
    \begin{itemize}
        \item Why this paper: Because we plan to use GA as one of the methods of optimization, which what the paper used,
        even though the paper's target wasn't heterogeneous systems.
        \item Paper research problem: General-purpose OS task scheduling problem.
        \item Paper goal: Optimize scheduling parameters using GA.
        \item Conclusion: GA’s can provide a highly flexible and user-friendly, near optimal solution to the general job sequencing problem.
    \end{itemize}

    \subsection*{[3] Ashish Venkat and Dean M Tullsen. Harnessing isa diversity: Design  of  a  heterogeneous-isa  chip  multiprocessor. In 2014 ACM/IEEE 41st International Symposium on Computer Architecture(ISCA), pages 121–132. IEEE, 2014.}
    \begin{itemize}
        \item Why this paper: To understand heterogeneous systems structures.
        \item Paper research problem: Which one is more power effective, heterogeneous or homogeneous systems?
        \item Paper goal: prove heterogeneous systems are more power efficient
        \item Conclusion: heterogeneous systems improves energy efficiency over the most efficient single-ISA design by 23\%.
    \end{itemize} 
\end{document}


